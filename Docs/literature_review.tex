\documentclass[a4paper]{article}

\usepackage[english]{babel}
\usepackage[utf8]{inputenc}
\usepackage{amsmath}
\usepackage{graphicx}
\usepackage{epsfig}
\usepackage[colorinlistoftodos]{todonotes}
\usepackage[hidelinks]{hyperref}
\usepackage[margin=1.3in]{geometry}

\title{Literature Review of Machine Learning Feature Selection Methods with emphasis on Medical Application on Breast Cancer}

\author{Thony Price \and Niklas Lindqvist}

\date{March 2018}

\begin{document}
\maketitle

\begin{abstract}

This document will provide a review of the past work connected to the topic of our Degree Project in Computer Science, Feature Selection Methods for Classification of Breast Cancer. It highlight different efforts of previous research that will inspire our project and be cited while making claims concerning the topic.

\end{abstract}

\section{Introduction}

The purpose of this litterature review is to build a foundation of knowledge for our upcomming project. The more we know about the topic beforehand the easier it will be to choose a relevant research question, refer to relevant material in out report as well as navigate the present research in the field. This document will also easily convert to a chapter about previous research.
\\\par

\noindent The structure of the document:
\begin{itemize} \itemsep0pt \parskip0pt \parsep0pt
	\item \textbf{Title:} Litterature title and reference.
	\item \textbf{Notes:} Key points from text.
	\item \textbf{Usage:} Potential use to us.
\end{itemize}

\section{Litterature}

\textbf{Title:} Machine learning for medical diagnosis: history, state of the art and perspective \cite{kononenko2001}.
\\
\textbf{Notes:} An overview of the development of intelligent data analysis in medicine from a machine learning perspective: a historical view, a state-of-the-art view, and a view on some future trends in this subfield of applied artificial intelligence.
\\
\textbf{Usage:} The paper goes far back and covers the outlook for Machine Learning in the 60's to state of the art. This overview of how this dicipline evolved is good to present in our project too.
\\\par

\noindent
\textbf{Title:} Improve Computer-Aided Diagnosis With Machine Learning Techniques Using Undiagnosed Samples \cite{li2007}.
\\
\textbf{Notes:}
\\
\textbf{Usage:}
\\\par

\noindent
\textbf{Title:} Global trends in breast cancer incidence and mortality 1973–1997 \cite{althuis2005}.
\\
\textbf{Notes:}
\\
\textbf{Usage:}
\\\par

\noindent
\textbf{Title:} Beyond randomized controlled trials - Organized mammographic screening substantially reduces breast carcinoma mortality \cite{tabar2001}.
\\
\textbf{Notes:}
\\
\textbf{Usage:}
\\\par

\noindent
\textbf{Title:} Discovering Mammography-based Machine Learning Classifiers for Breast Cancer Diagnosis \cite{ramos2012}.
\\
\textbf{Notes:}
\\
\textbf{Usage:}
\\\par

\noindent
\textbf{Title:} Classifier ensemble construction with rotation forest to improve medical diagnosis performance of machine learning algorithms \cite{akin2011}.
\\
\textbf{Notes:}
\\
\textbf{Usage:}
\\\par

\noindent
\textbf{Title:} Support vector machines combined with feature selection for breast cancer diagnosis \cite{akay2009}.
\\
\textbf{Notes:}
\\
\textbf{Usage:}
\\\par

\noindent
\textbf{Title:} A comparative study on the effect of feature selection on classification accuracy \cite{karabulut2012}.
\\
\textbf{Notes:}
\\
\textbf{Usage:}
\\\par

\noindent
\textbf{Title:} MUC-4 Evaluation metrics \cite{muc1992}.
\\
\textbf{Notes:}
\\
\textbf{Usage:}
\\\par

\noindent
\textbf{Title:} {UCI} Machine Learning Repository \cite{dua:2017}.
\\
\textbf{Notes:}
\\
\textbf{Usage:}
\\\par

\noindent
\textbf{Title:} Radiographer involvement in mammography image interpretation: A survey of United Kingdom practice \cite{culpan2016}.
\\
\textbf{Notes:}
\\
\textbf{Usage:}
\\\par

\noindent
\textbf{Title:} Breast Cancer in Low- and Middle-Income Countries: Why We Need Pathology Capability to Solve This Challenge \cite{martei2018}.
\\
\textbf{Notes:}
\\
\textbf{Usage:}
\\\par

\noindent
\textbf{Title:} An introduction to variable and feature selection \cite{guyon2003}.
\\
\textbf{Notes:}
\\
\textbf{Usage:}
\\\par

\noindent
\textbf{Title:} Diagnostic value of fine-needle aspiration biopsy for breast mass: a systematic review and meta-analysis \cite{Yu2012}.
\\
\textbf{Notes:} Fine needle aspiration (FNA) is a minimal invasive yet maximally diagnostic method regarding breast cancer. Compared to open biopsy the method is cheap to preform and its results can be provided within a shorter time. FNA is part of the triple- diagnostic method that reduces the risk to missclassify a cancer diagnosis to \textless 0.1\%
\\
\textbf{Usage:}
\\\par



\subsection{Conclusion}

Here is a conclusion!

\bibliographystyle{acm}
\bibliography{../LaTeX/references.bib}
\end{document}
