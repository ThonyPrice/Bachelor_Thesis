\chapter{Introduction}

Breast cancer is a disease of major concern and is the leading cause of cancer deaths among women \parencite{althuis2005}. At present there are no effective ways to prevent breast cancer. However, efficient diagnosis in an early stage can increase the chance of full recovery. This makes early detection and diagnosis an important issue and screening mammography is the primary imaging modality for early detection of breast cancer \parencite{tabar2001}.

Hospitals today collect data to do monitoring and some data such as mammography screenings can be collected and shared in information systems. The data can be used by medical personnel to increase their understanding of different diseases. It can also be used in computer aided diagnostic (CAD) where machine learning algorithms enable tools for intelligent data analysis. CAD makes use of machine learning techniques that learn a hypothesis, a statistical prediction about a patient's diagnosis from a large set of previously diagnosed examples.  The overarching purpose is to assist medical experts in more efficient and accurate diagnostics \parencite{li2007}. Machine learning algorithms is a well studied field within medical diagnosis and well suited for analyzing medical data, especially within small specialized diagnostic problems such as breast cancer \parencite{kononenko2001}.

Multiple studies of CAD on breast cancer have been conducted, primarily focusing on classifying mammography data as malignant or not, such those of \textcite{ramos2012} and \textcite{akay2009}. The act of feature selection, removing redundant or irrelevant features from a dataset, can provide classifiers to be faster, more cost-effective and accurate [8]. It is also explicitly mentioned as a topic in need of more research in studies on breast cancer diagnostics \parencite{akin2011}.


\section{Research Question}

In our thesis we will study the impact of different feature selection methods on the classification rate of malignant breast cancer by different machine learning methods. We aim to answer the following:

\begin{itemize}
  \item Does the feature selection improve the accuracy of classification compared to using all features?
  \item Is the effect of feature selection dependent on the classification method?
\end{itemize}

Our hypothesis is that overall the feature selection will improve the classification rate of the machine learning methods used in the context of breast cancer classification. This hypothesis is based on previous research reported by \textcite{karabulut2012}, where it was found that classification accuracy on 15 different datasets of medical and non-medical data was increased by the use of filtering methods for feature selection.

Our research differs from the work presented in \parencite{karabulut2012} in the amount of datasets used and amount of classification methods. In our project we will put more emphasis on breast cancer using only datasets of that type. The previous work only investigated feature selection methods by filtering which we will extend by implementing wrapper methods. Lastly, the research scope in this thesis includes a study of the effect of different feature selection methods on Support Vector Machines (SVM), which was not discussed by \parencite{karabulut2012}.


\section{Approach}

Trials will be conducted with feature selection by using both Wrapper methods and feature selection Filter methods. The result of the feature selection methods will be used with different classifiers to evaluate their performance. To broaden the base for comparison we will use several classifiers, Decision Tree (DT) a logic based algorithms, Artificial Neural Network (ANN) a perceptron-based technique, Naïve Bayes (NB) a statistical learning algorithms and Support Vector Machines (SVM) \parencite{wallace2007}. These classifiers are also commonly used in CAD \parencite{ramos2012}, \parencite{akay2009}, \parencite{li2007}. Feature selection (FS) methods and classifiers are denoted in table \ref{table:methods}.

\begin{table}[ht]
\begin{center}
\begin{tabular}{ l | l }
\multicolumn{2}{ c }{\textbf{Implementations}} \\
\hline
\multirow{4}{*}{Classifiers}
 & Artificial Neural Network (ANN) \\
 & Decision Tree (DT) \\
 & Naïve Bayes (NB) \\
 & Support Vector Machine (SVM) \\ \hline
\multirow{2}{*}{FS by Wrapping}
 & Sequential Backward FS (SBS) \\
 & Sequential Forward FS (SFS) \\ \hline
\multirow{2}{*}{FS by Filtering}
 & Chi-square (Chi2) \\
 & Entropy \\
\hline
\end{tabular}
\caption{All feature selection methods will be applied to each classifier  in the implementations}
\label{table:methods}
\end{center}
\end{table}

A comparison of the classification rate on the machine learning methods without using any feature selection and classification rate using feature selection will be conducted in order to establish the importance of feature selection in different machine learning approaches when classifying breast cancer. The evaluation of impact by FS will be measured by computing the ratio between best accuracy achieved with and without using feature selection. Several datasets with different types of features will be evaluated. Details on the datasets will be presented in section \ref{sec:Datasets}


\section{Outline}

In the Background section earlier work on the topic will be more thoroughly presented on order to clarify how this thesis relates to the field. The following chapter, Methods will detail the methods used to achieve the results that are presented in chapter 4. Chapter 5 seeks to discuss the impact of the results on a larger scale. Finally, chapter 6 draws conclusion based on the analysis and answers our research questions.
