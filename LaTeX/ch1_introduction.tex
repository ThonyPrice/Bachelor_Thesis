\chapter{Introduction}

Breast cancer is a disease of major concern and is the leading cause of cancer deaths among women \parencite{althuis2005}. At present there are no effective ways to prevent breast cancer. However, efficient diagnosis in an early stage can increase the chance of full recovery. This makes early detection and diagnosis an important issue where currently mammography screenings is the primary imaging modality for early detection of breast cancer \parencite{tabar2001}.

Hospitals today collect data to do monitoring and some of this data, such as mammography screenings can be collected and shared in information systems. The data can be used by medical personnel to increase their understanding of different diseases. It can also be used in computer aided diagnostics (CAD) where machine learning algorithms enable tools for intelligent data analysis. CAD makes use of machine learning techniques that learn a hypothesis, a statistical prediction about a patient's diagnosis from a large set of previously diagnosed examples.  The overarching purpose is to assist medical experts in more efficient and accurate diagnostics \parencite{li2007}. Machine learning algorithms is a well studied field within medical diagnosis and well suited for analyzing medical data, especially within small specialized diagnostic problems such as breast cancer \parencite{kononenko2001}.

Multiple studies of CAD on breast cancer have been conducted, primarily focusing on classifying mammography data of tumors as malignant or not, such those of \textcite{ramos2012} and \textcite{akay2009}. The act of feature selection, removing redundant or irrelevant features from a dataset, can provide classifiers to be faster, more cost-effective and accurate. With feature selection the understandability can be improved which is a clear benefit when it comes to medical decisions \parencite{karabulut2012}. It is also explicitly mentioned as a topic in need of more research in studies made on breast cancer diagnostics \parencite{akin2011}.


\section{Research Question}

In our thesis we will study the impact of four feature selection methods on the classification rate of malignant breast cancer by four different machine learning methods. We aim to answer the following:

\begin{itemize}
  \item Does the feature selection improve the accuracy of classification compared to using all features?
  \item Is the effect of feature selection dependent on the classification method?
\end{itemize}

Our hypothesis is that overall the feature selection will improve the classification rate of the machine learning methods used in the context of breast cancer classification. This hypothesis is based on previous research reported by \textcite{karabulut2012}, where it was found that classification accuracy on 15 different datasets of medical and non-medical data was increased by the use of filtering methods for feature selection.

Our research differs from the work presented in \parencite{karabulut2012} in the amount of datasets used and number of classification methods. In our project we will put more emphasis on breast cancer using only datasets of that type. The previous work only investigated feature selection methods by filtering which we will extend by implementing wrapper methods. Lastly, the research scope in this thesis includes a study of the effect of different feature selection methods on Support Vector Machines (SVMs), which was not included in \parencite{karabulut2012}.


\section{Approach}

Trials will be conducted with feature selection by using both wrapper methods and feature selection filter methods. The result of the feature selection methods will be used with different classifiers to evaluate their performance. To broaden the base for comparison we will use several classifiers, Decision Tree (DT) a logic based algorithms, Artificial Neural Network (ANN) a perceptron-based technique, Naïve Bayes (NB) a statistical learning algorithms and Support Vector Machines (SVM) \parencite{wallace2007}. These classifiers are commonly used in CAD and thus relevant to study \parencite{ramos2012}, \parencite{akay2009}, \parencite{li2007}. Feature selection (FS) methods and classifiers included in this report are denoted in table \ref{table:methods}.

\begin{table}[ht]
\begin{center}
\begin{tabular}{ l | l }
\multicolumn{2}{ c }{\textbf{Included classifiers and FS-methods}} \\
\hline
\multirow{4}{*}{Classifiers}
 & Artificial Neural Network (ANN) \\
 & Decision Tree (DT) \\
 & Naïve Bayes (NB) \\
 & Support Vector Machine (SVM) \\ \hline
\multirow{2}{*}{FS by Wrapping}
 & Sequential Backward FS (SBS) \\
 & Sequential Forward FS (SFS) \\ \hline
\multirow{2}{*}{FS by Filtering}
 & Chi-square (Chi2) \\
 & Entropy \\
\hline
\end{tabular}
\caption[]
{\small All classifiers and feature selection methods included in this paper. Each classifier will be tested with each FS-method yielding 16 distinct combinations.}
\label{table:methods}
\end{center}
\end{table}


A comparison between the classification rate of the machine learning methods without using any feature selection, and classification rate when using feature selection will be conducted. The comparison may then establish the importance of feature selection in different machine learning approaches when classifying breast cancer. The evaluation of impact by FS will be measured by computing the ratio between best accuracy achieved with and without using feature selection. Several datasets with different types of features will be used for evaluation. Details on the datasets will be presented in section \ref{sec:Datasets}. The reason for using multiple datasets is to strengthen the basis for statistical evaluation and possibility conclude a more generalized result.


\section{Scope}

The scope of this thesis is limited by the amount of datasets, classifiers and FS-methods included. Having four of each, the aim is to conclude a result that generalize well. However, it must be considered these are a small selection of all the available possibilities. The data represents a few of the possible ways to collect data when making breast cancer diagnostics. There are many more classifiers and each can be tuned into countless of configurations. Regarding the feature selection methods we evaluate two filter methods and two wrapper methods, it exists many more and also embedded methods which is not included in out thesis.

These limitations results in constraining our scope to the classifiers and FS-methods presented in table \ref{table:methods} and the datasets described in \ref{sec:Datasets}.
