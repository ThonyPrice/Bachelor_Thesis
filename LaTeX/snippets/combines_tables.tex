\begin{table}[!htb]

    \noindent\makebox[\textwidth]{%

      \begin{subtable}{0.72\linewidth}
        \centering
          \begin{tabular}{|l|l|l|l|l|l}
\cline{1-5}
        \textbf{ANN} & MIAS              & EN                & RHH               & WBCD      &         \\
\cline{1-5}
Chi2    & 0.58 & 0.59 & 0.64 & 0.71 \\
Entropy & 0.56 & \textbf{0.84} & \textbf{0.90} & 0.62 \\
SBS     & 0.54 & 0.55 & 0.59 & \textbf{0.74} \\
SFS     & \textbf{0.59} & 0.51 & 0.59 & 0.68 \\
Full    & 0.57 & 0.68 & 0.60 & 0.53 \\
\cline{1-5}
Gain    & 0.04 & 0.24 & 0.51 & 0.41 & 1.2 \\
\cline{1-5}
\end{tabular}

          \caption[]
          \small{}
          \label{table:ANN}
      \end{subtable}%
      \begin{subtable}{0.72\linewidth}
        \centering
          \begin{tabular}{|l|l|l|l|l|l}
\cline{1-5}
        \textbf{CART} & MIAS              & EN                & RHH               & WBCD      &         \\
\cline{1-5}
Chi2    & 0.70           & 0.69           & 0.90           & \textbf{0.97}  &         \\
Entropy & 0.53           & \textbf{0.83}  & 0.91           & 0.96           &         \\
SBS     & 0.63           & 0.67           & 0.91           & 0.95           &         \\
SFS     & 0.67           & 0.67           & \textbf{0.92}  & \textbf{0.97}  &         \\
Full    & \textbf{0.77}  & 0.69           & 0.90           & 0.94           &         \\
\cline{1-5}
\cline{1-5}
Gain    & -0.09           & 0.21           & 0.03           & 0.03           & 0.18 \\
\cline{1-5}
\end{tabular}

          \caption[]
          \small{}
          \label{table:CART}
      \end{subtable}
    }

    \vskip
    \baselineskip

    \noindent\makebox[\textwidth]{%

      \begin{subtable}{0.72\linewidth}
        \centering
          \begin{tabular}{|l|l|l|l|l|l}
\cline{1-5}
        & MIAS              & EN                & RHH               & WBCD      &         \\
\cline{1-5}
Chi2    & \textbf{0.77}  & \textbf{0.74}  & \textbf{0.94}  & 0.96           &         \\
Entropy & \textbf{0.77}  & 0.55           & 0.91           & 0.97           &         \\
SBS     & 0.43           & 0.71           & 0.91           & \textbf{0.97}  &         \\
SFS     & 0.43           & 0.71           & 0.91           & \textbf{0.97}  &         \\
Full    & \textbf{0.77}  & 0.66           & \textbf{0.94}  & 0.96           &         \\
\cline{1-5}
\cline{1-5}
Gain    & 0                 & 0.12           & 0                 & 0.01           & 0.14 \\
\cline{1-5}
\end{tabular}

          \caption[]
          \small{}
          \label{table:NB}
      \end{subtable}%
      \begin{subtable}{0.72\linewidth}
        \centering
          \begin{tabular}{|l|l|l|l|l|l}
\cline{1-5}
        \textbf{SVM} & MIAS              & EN                & RHH               & WBCD      &         \\
\cline{1-5}
Chi2    & \textbf{0.57}  & 0.73           & 0.90           & 0.63           &         \\
Entropy & 0.33           & \textbf{0.83}  & \textbf{0.91}  & 0.63           &         \\
SBS     & 0.53           & 0.68           & 0.88           & \textbf{0.93}  &         \\
SFS     & 0.53           & 0.68           & 0.88           & \textbf{0.93}  &         \\
Full    & \textbf{0.57}  & 0.73           & 0.90           & 0.61           &         \\
\cline{1-5}
\cline{1-5}
Gain    & 0                 & 0.14           & 0.02           & 0.51           & 0.68 \\
\cline{1-5}
\end{tabular}

          \caption[]
          \small{}
          \label{table:SVM}
      \end{subtable}
    }

    \caption[]
    {\small
      Mean accuracy of 10-fold cross-validation accuracies achieved on each dataset when applying some FS-method or with Full dataset. Tables are categorized by classifier; (a) ANN, (b) CART Decision tree, (c) Na\"ive Bayes and (d) Support Vector Machine. The highest accuracy is highlighted by bold font and Gain represents ratio between the best accuracy achieved by FS and the accuracy by Full dataset. Accumulated Gain is last on the Gain row. Best accuracy by FS is equivalent or greater than with Full dataset in every instance except in (b) with CART classifier on MIAS dataset.
    }

\end{table}
