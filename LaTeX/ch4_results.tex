\chapter{Results and analysis}

We performed classification with four different classifiers on four different datasets. In each dataset-classifier combination we measured classification accuracy when applying four different feature selection (FS) methods. The aim is to investigate the impact of feature selection, and the interaction between different classifiers and FS-methods. To build a basis for comparisons, classification accuracy was measured without FS too.

\section{Variation among factors of experiments}
\label{Variation_among_factors}

Before analyzing the results we consider potential interactions between different factors involved in the experiment, these are:

\begin{enumerate}
  \item Datasets
  \item Machine Learning classifiers
  \item Feature selection methods
\end{enumerate}

Having four of each factor, there are 64 distinct combinations. In our scope we are particular interested in the interactions Classifiers:FS-methods. However, we will begin with looking at Datasets:FS-methods to evaluate the datasets influence on the classification accuracy.

\subsection{Datasets with larger dimensionality achieves better classification accuracy with feature selection}

Studying the interaction between datasets and FS-methods we regarded these factors as independent variables (IVs). Each IV combination is represented by four values, the best accuracy achieved with current FS-method by each of the four classifiers.

Two-way analysis of variance (ANOVA) is performed on the IVs. The results are presented in \ref{table:anova_values_data} and suggests three conclusions:

% ANOVA computations, dataset-fs
\begin{table}[ht]
  \begin{center}
  \begin{tabular}{l|r|r|r|r|l}
  \cline{2-5}
  & $RSS$ & $df$ & $F$ & $P(>F)$ \\ \cline{1-5}
  \multicolumn{1}{ |l| }{\textbf{Dataset}}
  & 0.7383 &  3.0 & 15.939 & 2.5241e-07 & *** \\
  \cline{1-5}
  \multicolumn{1}{ |l| }{\textbf{Method}}
  & 0.0099 &  3.0 & 0.2137 & 8.8645e-01 & \\
  \cline{1-5}
  \multicolumn{1}{ |l| }{\textbf{Dataset:Method}}
  & 0.0627 &  9.0 & 0.4514 & 8.9938e-01 & \\
  \cline{1-5}
  \multicolumn{1}{ |l| }{\textbf{Residual}}
  & 0.7412 &  48.0 \\ \cline{1-3}
  \end{tabular}
  \caption[]%
  {{\small ANOVA values of accuracy in relation to dataset, method and the interaction of datasets and methods. $RSS$: Residual sum of squares. $df$: Degrees of Freedom. $F$: Mean Square for the Model divided by the Mean Square for Error (error/residual).  $P(>F)$: the significance probability associated with $F$. The stars indicate the range of significant level: 0 "***" 0.001 "**" 0.01 "*" 0.05 " . " 0.1 " " 1.}}
  \label{table:anova_values_data}
  \end{center}
\end{table}


First, the significance of Dataset allows to conclude a dependency between accuracies and datasets is present. This is to be expected because different datasets provide varying basis for prediction. However, as all accuracies used in the test was obtained by feature selection this test gives no information on whether the correlation lies among the datasets themselves of the application of feature selection.

Secondly, the absence of significance in Method mean we can not reject that there is no correlation between FS-method and accuracy.

Thirdly, no significance of Dataset:Method indicates we can not reliably conclude that the interaction between datasets and methods has a correlation with achieved accuracy.

Visualizing the values used in the ANOVA test, the correlation between datasets and accuracy is apparent \ref{fig:comp_acc_datasets}. It is positive in respect to number of attributes suggesting either feature selection has a larger impact on classification accuracy when data has a higher dimensionality or higher dimensioned datasets on their own lead to better accuracy.

\begin{figure*}[ht]
  \centering
    \begin{subfigure}[b]{0.475\textwidth}
        \centering
        \includegraphics[width=\textwidth]{../plots_with_std_fill/comp_acc_datasets.png}
        \caption[]%
        {{\small Datasets and accuracy displays positive correlation independently of applied FS method.}}
        \label{fig:comp_acc_datasets}
    \end{subfigure}
  \hfill
    \begin{subfigure}[b]{0.475\textwidth}
        \centering
        \includegraphics[width=\textwidth]{../plots_with_std_fill/comp_classif_datasets.png}
        \caption[]%
        {{\small Some correlation between different classifiers and FS-methods is evident.}}
        \label{fig:comp_classif_datasets}
    \end{subfigure}
  \caption[]
  {\small Correlation between factors of the experiments.}
  \label{fig:anova_plots}
\end{figure*}


\subsection{Interaction between FS-methods and classifiers}
\label{sec:fs_methods_classifiers}

To investigate the effect of feature selection on different classifiers we performed two-way ANOVA. The values constitutes of each accuracy achieved of all distinct classifier and FS-method on all datasets. The ANOVA result is presented in \ref{table:anova_values_classif}.

\begin{table}[ht]
  \begin{center}
  \begin{tabular}{l|r|r|r|r|l}
  \cline{2-5}
  & $RSS$ & $df$ & $F$ & $P(>F)$ \\ \cline{1-5}
  \multicolumn{1}{ |l| }{\textbf{Classifier}}
  & 0.3336 &  3.0 & 4.660   & 0.00615    & ** \\
  \cline{1-5}
  \multicolumn{1}{ |l| }{\textbf{Method}}
  & 0.0099 &  3.0 & 0.138   & 0.93666 & \\
  \cline{1-5}
  \multicolumn{1}{ |l| }{\textbf{Classifier:Method}}
  & 0.0631 &  9.0 & 0.294   & 0.97314 \\
  \cline{1-5}
  \multicolumn{1}{ |l| }{\textbf{Residual}}
  & 1.1455 &  48.0 \\ \cline{1-3}
  \end{tabular}
  \caption[]%
  {{\small ANOVA values of accuracy in relation to classifier, method and the interaction of classifiers and methods. $RSS$: Residual sum of squares. $df$: Degrees of Freedom. $F$: Mean Square for the Model divided by the Mean Square for Error (error/residual).  $P(>F)$: the significance probability associated with $F$. The stars indicate the range of significant level: 0 "***" 0.001 "**" 0.01 "*" 0.05 " . " 0.1 " " 1.}}
  \label{table:anova_values_classif}
  \end{center}
\end{table}


The significance of both Classifier and Method concludes they independently effect what accuracy is achieved. Their interaction, Classifier:Method is significant too and allows to reason there is a dependency between accuracy and the interaction between classifiers and methods. Plotting the values used in the test visualizes this correlation \ref{fig:comp_classif_datasets}.

To further analyze the interaction of classifiers and FS-methods we performed post-hoc analysis by Tukey's test \parencite{Haynes2013}. The significant results \ref{table:Tukey_values} show the difference among the groups lies mainly in which classifier is used.


The result indicates that wrapper methods has a superior classification accuracy compared to filter methods in some cases. In table \ref{table:t_test_result} the classifiers who benefits from certain wrapper methods is compared to filter methods.

The groups in Tukey's test which are significantly different while using same classifier are summarized in \ref{table:t_test_result}. The diff indicated mean difference in accuracy between the first- and second group. From this we can conclude wrapper methods has a superior classification accuracy compared to filter methods in some cases. These cases concern the classifiers CART, NB and SVM.

\begin{table}[htbp]
\begin{center}
\begin{tabular}{|l|c|c|c|c|l}
\cline{1-5}
\textbf{Classif:method} & \textbf{diff} & \textbf{lwr} & \textbf{upr} & \textbf{p} & \\
\cline{1-5}
           NB:sfs-NB:chi2 &   0.35 &   0.14 &   0.56 &  0.000 &  *** \\
         SVM:sbs-SVM:chi2 &   0.33 &   0.12 &   0.54 &  0.000 &  *** \\
       CART:sfs-CART:chi2 &   0.31 &   0.10 &   0.52 &  0.000 &  *** \\
           NB:sbs-NB:chi2 &   0.31 &   0.10 &   0.52 &  0.000 &  *** \\
        NB:sfs-NB:entropy &   0.28 &   0.07 &   0.49 &  0.001 &   ** \\
       CART:sbs-CART:chi2 &   0.28 &   0.07 &   0.49 &  0.002 &   ** \\
        NB:sbs-NB:entropy &   0.24 &   0.03 &   0.45 &  0.010 &    * \\
    CART:sfs-CART:entropy &   0.23 &   0.02 &   0.44 &  0.022 &    * \\
         SVM:sfs-SVM:chi2 &   0.21 &   0.00 &   0.42 &  0.049 &    * \\
    CART:sbs-CART:entropy &   0.20 &  -0.01 &   0.41 &  0.090 &    . \\
\cline{1-5}
\end{tabular}
\caption[]
{\small
  Significant values of Tukey's test where classifiers are the same. diff shows wrapper methods (sfs and sbs) perform at least 20\% better than filter methods (chi2 and entropy). The increased performance is present in classifiers CART, NB and SVM. diff: Difference between means. lwr: Lower bound of confidence interval. upr: Upper bound of confidence interval. p: Significance probability. The stars indicate the range of significant level: 0 "***" 0.001 "**" 0.01 "*" 0.05 " . " 0.1 " " 1.
}
\label{table:t_test_result}
\end{center}
\end{table}


To conclude these results, they suggest one primarily has to carefully select the classifier as this factor has the largest impact on classification rate. If the classifier is NB or CART and the goal is to achieve a high classification accuracy SBS or SFS should be preferred over Chi2 and Entropy. If SVM is used, SBS is preferred over Chi2 and Entropy with regards to classification rate.

\section{Classification improvements}

After performing classification with and without FS-methods, all results of each classifier was collected. Results are presented in tables for ANN \ref{table:ANN}, CART \ref{table:CART}, NB \ref{table:NB} and SVM \ref{table:SVM} where the highest achieved accuracy is highlighted in bold format. The improvement is measured in gain, it is the ratio between best achieved FS accuracy and full dataset accuracy.

% --- Tables ---
\begin{table}[!htb]

    \noindent\makebox[1.05\textwidth]{%

      \begin{subtable}{0.73\linewidth}
        \centering
          \begin{tabular}{|l|l|l|l|l|l}
\cline{1-5}
        & MIAS              & EN                & RHH               & WBCD      &         \\
\cline{1-5}
Chi2    & \textbf{0.63333}  & 0.55357           & 0.66993           & \textbf{0.72810}  &         \\
Entropy & 0.56667           & \textbf{0.83214}  & \textbf{0.90196}  & 0.63762           &         \\
SBS     & 0.50000           & 0.55214           & 0.66144           & 0.64286           &         \\
SFS     & 0.53333           & 0.50595           & 0.65915           & 0.67381           &         \\
Full    & 0.60000           & 0.52714           & 0.57320           & 0.64143 \\
\cline{1-5}
\cline{1-5}
Gain    & 0.05555           & 0.57859           & 0.57355           & 0.13512           & 1.34281 \\
\cline{1-5}
\end{tabular}

          \caption[]
          \small{}
          \label{table:ANN}
      \end{subtable}%
      \begin{subtable}{0.73\linewidth}
        \centering
          \begin{tabular}{|l|l|l|l|l|l}
\cline{1-5}
        & MIAS              & EN                & RHH               & WBCD      &         \\
\cline{1-5}
Chi2    & 0.70000           & 0.69262           & 0.90196           & \textbf{0.96524}  &         \\
Entropy & 0.53333           & \textbf{0.83214}  & 0.91373           & 0.95905           &         \\
SBS     & 0.63333           & 0.67286           & 0.91307           & 0.95143           &         \\
SFS     & 0.66667           & 0.67286           & \textbf{0.91895}  & \textbf{0.96524}  &         \\
Full    & \textbf{0.76667}  & 0.68786           & 0.89608           & 0.93714           &         \\
\cline{1-5}
\cline{1-5}
Gain    & -0.0870           & 0.20975           & 0.02552           & 0.02998           & 0.17825 \\
\cline{1-5}
\end{tabular}

          \caption[]
          \small{}
          \label{table:CART}
      \end{subtable}
    }

    \vskip
    \baselineskip

    \noindent\makebox[1.05\textwidth]{%

      \begin{subtable}{0.73\linewidth}
        \centering
          \begin{tabular}{|l|l|l|l|l|l}
\cline{1-5}
        \textbf{NB} & MIAS              & EN                & RHH               & WBCD      &         \\
\cline{1-5}
Chi2    & \textbf{0.75} &  0.75 &  0.93 & 0.97 \\
Entropy & 0.72 &  0.55 &  0.93 & \textbf{0.98} \\
SBS     & 0.51 &  0.72 &  0.93 & 0.97 \\
SFS     & 0.51 &  0.72 &  0.93 & 0.97 \\
Full    & 0.57 &  \textbf{0.93} &  \textbf{0.96} & 0.96 \\
\cline{1-5}
Gain    & 0.30 & -0.19 & -0.03 & 0.02 & 0.1\\
\cline{1-5}
\end{tabular}

          \caption[]
          \small{}
          \label{table:NB}
      \end{subtable}%
      \begin{subtable}{0.73\linewidth}
        \centering
          \begin{tabular}{|l|l|l|l|l|l}
\cline{1-5}
        \textbf{SVM} & MIAS              & EN                & RHH               & WBCD      &         \\
\cline{1-5}
Chi2    &  0.59 &  0.75 & 0.92 & 0.66 \\
Entropy &  0.59 &  0.84 & 0.92 & 0.66 \\
SBS     &  0.59 &  0.75 & \textbf{0.94} & \textbf{0.94} \\
SFS     &  0.56 &  0.75 & 0.90 & \textbf{0.94} \\
Full    &  \textbf{0.74} &  \textbf{0.90} & 0.64 & 0.64 \\
\cline{1-5}
Gain    & -0.20 & -0.07 & 0.46 & 0.45 & 0.64 \\
\end{tabular}

          \caption[]
          \small{}
          \label{table:SVM}
      \end{subtable}
    }

    \caption[]
    {\small
      Mean accuracy of 10-fold cross-validation accuracies achieved on each dataset when applying some FS-method or with Full dataset. Tables are categorized by classifier; (a) ANN, (b) CART Decision tree, (c) Na\"ive Bayes and (d) Support Vector Machine. The highest accuracy is highlighted by bold font and Gain represents ratio between the best accuracy achieved by FS and the accuracy by Full dataset. Accumulated Gain is last on the Gain row. Best accuracy by FS is equivalent or greater than with Full dataset in every instance except in (b) with CART classifier on MIAS dataset.
    }
    \label{table:combines_tables}
\end{table}


\subsection{Improvement is significant}

To ensure the suggestion that application of some FS-method improves classification accuracy t-test was performed. The t-test values constitutes of the accuracies achieved without feature selection as one distribution. The other are the best accuracies achieved with feature selection. The t-test values are summarized in \ref{table:t_test_values}.

\begin{table}[!htb]

    \noindent\makebox[\textwidth]{%

      \begin{subtable}{0.65\linewidth}
        \centering
          \begin{tabular}{|l|l|l|l|l|l}
          \cline{1-5}
          \textbf{FS}      & MIAS           & EN             & RHH            & WBCD           &         \\
          \cline{1-5}
          ANN     & 0.63           & 0.83           & 0.90           & 0.73           &         \\
          CART    & 0.67           & 0.83           & 0.92           & 0.97           &         \\
          NB      & 0.77           & 0.74           & 0.94           & 0.97           &         \\
          SFS     & 0.57           & 0.83           & 0.91           & 0.93           &         \\
          \cline{1-5}
          \end{tabular}
          \caption[]
          \small{}
      \end{subtable}%
      \begin{subtable}{0.65\linewidth}
        \centering
          \begin{tabular}{|l|l|l|l|l|l}
          \cline{1-5}
          \textbf{Full}    & MIAS           & EN             & RHH            & WBCD           &         \\
          \cline{1-5}
          ANN     & 0.60           & 0.53           & 0.57           & 0.64           &         \\
          CART    & 0.77           & 0.69           & 0.90           & 0.94           &         \\
          NB      & 0.77           & 0.66           & 0.94           & 0.96           &         \\
          SFS     & 0.57           & 0.73           & 0.90           & 0.61           &         \\
          \cline{1-5}
          \end{tabular}
          \caption[]
          \small{}
      \end{subtable}
    }

    \caption[]%
    {\small
      (a) The highest accuracy achieved by any FS-method on all classifier-dataset combinations. (b) corresponding accuracies achieved without any feature selection on Full dataset. T-test comparing the distributions results in significantly higher results by classification with feature selection.}
    \label{table:t_test_values}

\end{table}



Performing the t-test a significant distinction between the distributions can not be drawn. This does not allows us to reject the null hypothesis that the distributions are equal. This implies that the conclusion that improvement of classification accuracy by feature selection can not be drawn.

\subsection{Differences among classifiers}

As addressed in section \ref{Variation_among_factors} classifiers behave differently depending on FS-method and datasets.
In order to compare classification improvement among different FS-methods and datasets \textbf{Gain} is computed. Gain measures the improvement of FS as the ratio between best achieved FS accuracy and full dataset accuracy. The accumulated gain for each classifier is presented in \ref{table:gain_comparison}. Below these results and differences are analyzed in further detail with respect to each classifier.

\begin{table}[hp]
  \begin{tabular}{|l|l|l|}
\hline
Ranking  & Classifier                & Accumulated gain  \\
\hline
1        & Artificial Neural Network & 1.34              \\
2        & Support Vector Machine    & 0.68              \\
3        & Decision Tree             & 0.18              \\
4        & Naive Bayes               & 0.14              \\
\hline
\end{tabular}

  \caption[]%
  {{\small Ranking of which classifiers gained most accuracy when comparing feature selection to full dataset.}}
  \label{table:gain_comparison}
\end{table}

\subsubsection{Artificial Neural Network}

Looking at the table \ref{table:gain_comparison} the accumulated gain was 1.34 which was the highest among all classifiers. However, ANN consistently performs the worst of all classifiers in terms of accuracy. The ANN also provides the least consistent results with strong fluctuations in the results and wide standard deviation margins. Such fluctuations may suggest issues regarding convergence. These characteristics are evident in plot \ref{fig:WBCD_chi2} showcasing the accuracy by each FS-method as a function of number of attributes..

\subsubsection{Support Vector Machine}

The SVM behaves very differently in respect to each dataset. Improvements are seen with a larger subset of attributes in the EN and MIAS dataset as seen in \ref{fig:EN_chi2} and \ref{fig:MIAS_sfs}. A negative trend on accuracy is observed on the WBCD datasets which might indicate an issue with dimensionality. In \ref{fig:RHH_sbs} an improved accuracy is evident with maximal accuracy achieved on a subset suggesting a positive effect of feature selection.

\subsubsection{Decision Tree}

The decision tree shows consistent performance generally increasing performance with an increased amount of features. Although in cases like plot \ref{fig:WBCD_chi2} and \ref{fig:MIAS_sbs} best accuracy is achieved with a subset of features displaying evident benefits of feature selection.

\subsubsection{Naïve Bayes}

Naïve Bayes had the least accumulated gain from feature selection as seen in table \ref{table:gain_comparison}. The largest improvement was seen in plot \ref{fig:EN_chi2} using 2 of 4 available attributes. In plots \ref{fig:WBCD_entropy} and \ref{fig:RHH_entropy} the accuracy presents little to no improvement when increasing the number of attributes.

\begin{figure*}[htp]
  \centering
  \begin{subfigure}[b]{0.475\textwidth}
      \centering
      \includegraphics[width=\textwidth]{../plots_with_std_fill/Cleaned_data_chi2_combined.png}
      \caption[]%
      {{\small Dataset EN using Chi2}}
      \label{fig:EN_chi2}
  \end{subfigure}
  \hfill
  \begin{subfigure}[b]{0.475\textwidth}
      \centering
      \includegraphics[width=\textwidth]{../plots_with_std_fill/data_FNA_chi2_combined.png}
      \caption[]%
      {{\small Dataset WBCD using Chi2}}
      \label{fig:WBCD_chi2}
  \end{subfigure}
  \vskip\baselineskip
  \begin{subfigure}[b]{0.475\textwidth}
      \centering
      \includegraphics[width=\textwidth]{../plots_with_std_fill/Data_mias_chi2_combined.png}
      \caption[]%
      {{\small Dataset MIAS using Chi2}}
      \label{fig:MIAS_chi2}
  \end{subfigure}
  \quad
  \begin{subfigure}[b]{0.475\textwidth}
      \centering
      \includegraphics[width=\textwidth]{../plots_with_std_fill/FNA_gb_chi2_combined.png}
      \caption[]%
      {{\small Dataset RHH using Chi2}}
      \label{fig:RHH_chi2}
  \end{subfigure}
  \caption[]
  {\small Combined plots of all datasets comparing each classifier when using chi2 for feature selection.}
  \label{fig:plots_chi2}
\end{figure*}

\begin{figure*}[htp]
  \centering
  \begin{subfigure}[b]{0.475\textwidth}
      \centering
      \includegraphics[width=\textwidth]{../plots_with_std_fill/Cleaned_data_entropy_combined.png}
      \caption[Network2]%
      {{\small Dataset EN using Entropy}}
      \label{fig:EN_entropy}
  \end{subfigure}
  \hfill
  \begin{subfigure}[b]{0.475\textwidth}
      \centering
      \includegraphics[width=\textwidth]{../plots_with_std_fill/data_FNA_entropy_combined.png}
      \caption[]%
      {{\small Dataset WBCD using Entropy}}
      \label{fig:WBCD_entropy}
  \end{subfigure}
  \vskip\baselineskip
  \begin{subfigure}[b]{0.475\textwidth}
      \centering
      \includegraphics[width=\textwidth]{../plots_with_std_fill/Data_mias_entropy_combined.png}
      \caption[]%
      {{\small Dataset MIAS using Entropy}}
      \label{fig:MIAS_entropy}
  \end{subfigure}
  \quad
  \begin{subfigure}[b]{0.475\textwidth}
      \centering
      \includegraphics[width=\textwidth]{../plots_with_std_fill/FNA_gb_entropy_combined.png}
      \caption[]%
      {{\small Dataset RHH using Entropy}}
      \label{fig:RHH_entropy}
  \end{subfigure}
  \caption[]
  {{\small Combined plots of all datasets comparing each classifier when using Entropy for feature selection.}}
  \label{fig:plots_entropy}
\end{figure*}

\begin{figure*}[htp]
  \centering
  \begin{subfigure}[b]{0.475\textwidth}
      \centering
      \includegraphics[width=\textwidth]{../plots_with_std_fill/Cleaned_data_sb_combined.png}
      \caption[]%
      {{\small}}
      \label{fig:EN_sbs}
  \end{subfigure}
  \hfill
  \begin{subfigure}[b]{0.475\textwidth}
      \centering
      \includegraphics[width=\textwidth]{../plots_with_std_fill/data_FNA_sb_combined.png}
      \caption[]%
      {{\small}}
      \label{fig:WBCD_sbs}
  \end{subfigure}
  \vskip\baselineskip
  \begin{subfigure}[b]{0.475\textwidth}
      \centering
      \includegraphics[width=\textwidth]{../plots_with_std_fill/Data_mias_sb_combined.png}
      \caption[]%
      {{\small}}
      \label{fig:MIAS_sbs}
  \end{subfigure}
  \quad
  \begin{subfigure}[b]{0.475\textwidth}
      \centering
      \includegraphics[width=\textwidth]{../plots_with_std_fill/FNA_gb_sb_combined.png}
      \caption[]%
      {{\small}}
      \label{fig:RHH_sbs}
  \end{subfigure}
  \caption[]
  {{\small a: Dataset EN using SBS, b: Dataset WBCD using SBS, c Dataset MIAS using SBS, d: Dataset RHH using SBS. Combined plots of all datasets comparing each classifier when using SBS for feature selection.}}
  \label{fig:plots_sbs}
\end{figure*}

\begin{figure*}[htbp!]
  \centering
  \begin{subfigure}[b]{0.475\textwidth}
      \centering
      \includegraphics[width=\textwidth]{../plots_with_std_fill/Cleaned_data_sf_combined.png}
      \caption[]%
      {{\small Dataset EN using SFS}}
      \label{fig:EN_sfs}
  \end{subfigure}
  \hfill
  \begin{subfigure}[b]{0.475\textwidth}
      \centering
      \includegraphics[width=\textwidth]{../plots_with_std_fill/data_FNA_sf_combined.png}
      \caption[]%
      {{\small Dataset WBCD using SFS}}
      \label{fig:WBCD_sfs}
  \end{subfigure}
  \vskip\baselineskip
  \begin{subfigure}[b]{0.475\textwidth}
      \centering
      \includegraphics[width=\textwidth]{../plots_with_std_fill/Data_mias_sf_combined.png}
      \caption[]%
      {{\small Dataset MIAS using SFS}}
      \label{fig:MIAS_sfs}
  \end{subfigure}
  \quad
  \begin{subfigure}[b]{0.475\textwidth}
      \centering
      \includegraphics[width=\textwidth]{../plots_with_std_fill/FNA_gb_sf_combined.png}
      \caption[]%
      {{\small Dataset RHH using SFS}}
      \label{fig:RHH_sfs}
  \end{subfigure}
  \caption[]
  {{\small Combined plots of all datasets comparing each classifier when using SFS for feature selection.}}
  \label{fig:plots_sfs}
\end{figure*}


\newpage
\section{Computation time}

\textbf{Under construction, profiling not run yet.}

Profiling the execution of running all experiments X\% of CPU-time was allocated to the Wrapper algorithms. As mentioned a finding the best possible subset of features is considered a NP-hard problem meaning a solution can not be found in polynomial time. This clearly suggest favouring filtering methods when choosing a feature selection method having limited computational resources.

\section{Source of errors}
\label{sec:source_of_errors}

There are two main factor that may risk propagate error into the results, libraries and datasets.

The libraries provide all functionality of the classifiers, filter selection methods and analysis tools. A major part of the implementation therefore constitutes of these libraries. A implementation built upon faulty methods can not provide any trustworthy results. The Scikit library \parencite{scikit-learn} is well renowned and widely used in both industry and research thus inspire confidence in its robustness. The mlxtend library is used for SBS and SFS methods and is still a open source with less coverage than Scikit \parencite{mlextend}. Still it has a active community and many release versions indicating it's well managed.

Because we use a multitude of datasets their differences may influence our results in a way that invalidates our findings. \textcite{c201416} claim that feature selection methods can only be compared on the same dataset. However our comparisons do not regard the performance of specific FS-methods but their impact at large. Also, these interactions is addressed in \ref{Variation_among_factors}.

Due to varying number of examples of the datasets where some are rather small, evaluation by kFold may infer very small test batches and lead to skewed results. The impact may be enlarged considering our classification is binary which means consistently predicting one class where that class is predominant archives non representable results for such predictor. This influence is minimized by using stratified kFold but as no check of distribution of classes in test batches was made this might be a source of error.

In analysis of the results two-way anova is performed to evaluate the effect of two factors on accuracy. As mentioned in \ref{Variation_among_factors} our experiments conclude of three factors leading us to regard one of these as independent to the accuracy. If this is not the case the results from analysis may suffer from bias of this parameter.
