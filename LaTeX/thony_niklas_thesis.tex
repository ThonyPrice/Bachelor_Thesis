\documentclass{kththesis}

\usepackage[linesnumbered,ruled]{algorithm2e}
\usepackage{amsmath}
\usepackage{booktabs}
\usepackage{biblatex}
\usepackage{chngpage}
\usepackage{csquotes} % Recommended by biblatex
\usepackage{hyperref}
\usepackage{graphicx}
\usepackage{mwe}
\usepackage{multirow}
\usepackage{numprint}
\usepackage[parfill]{parskip}
\usepackage{subcaption}

\addbibresource{references.bib} % The file containing our references, in BibTeX format


% --- TITLE & FRONT PAGE ---

\title{Feature Selection Methods for Classification of Breast Cancer}
\alttitle{Attributurvalsmetoder för klassificering av bröstcancer }
\author{Niklas Lindqvist\newline Thony Price}
\email{nlinq@kth.se\newline thonyp@kth.se}
\supervisor{Pawel Herman}
\examiner{Örjan Ekeberg}
\programme{Degree Project in Computer Science}
\school{EECS School}
\date{\today}


\begin{document}
% Frontmatter includes the titlepage, abstracts and table-of-contents
\frontmatter
\titlepage


% --- ABSTRACT ENGLISH ---

\begin{abstract}

  Breast cancer is the leading cause of cancer deaths among women today but computer aided diagnosis has proved efficient in assisting medical experts to set an early diagnosis improving the chance of recovery. Computer aided diagnostics utilises machine learning to make a prediction whether a patient has a benign or malignant cancer. In the process a patients data is put into machine learning algorithms and a classification is made. Applying feature selection the algorithms can be fed data with lower dimensionality and produce a more accurate result. We have found that all the classifiers used in this paper had a higher maximum classification accuracy when using feature selection.
\end{abstract}


% --- ABSTRACT SWEDISH ---

\begin{otherlanguage}{swedish}
  \begin{abstract}

  Bröstcancer är idag den ledande cancerformen som orsakar förtidig död hos kvinnor. Datordriven diagnostisering har visat sig effektiv i att assistera medicinska experter med att sätta en tidig diagnos för cancers och därmed öka chanserna för tillfrisknande hos patienten. Datordriven diagnostisering använder sig av maskininlärningsmetoder för att göra en prediktion huruvida en patient har god eller elakartad cancer. I denna processes används en patients data av maskininlärningsalgoritmen och en klassificering kan göras. Applikationen av attributurvalsmetoder innebar att algoritmen kan använda sig av data med färre dimensioner och producera ett mer träffsakert resultat. Vi fann att samtliga klassificerare som undersöks i denna rapport visade fördel med högre klassificeringsträffsäkerhet när attributurvalsmetoder användes.



  \end{abstract}

\end{otherlanguage}


\tableofcontents
% Mainmatter is where the actual contents of the thesis goes
\mainmatter


% --- CONTENT ---

% --- 1. INTRODUCTION CHAPTER ---
\chapter{Introduction}

Breast cancer is a disease of major concern and is the leading cause of cancer deaths among women \parencite{althuis2005}. At present there are no effective ways to prevent breast cancer. However, efficient diagnosis in an early stage can increase the chance of full recovery. This makes early detection and diagnosis an important issue where currently mammography screenings is the primary imaging modality for early detection of breast cancer \parencite{tabar2001}.

Hospitals today collect data to do monitoring and some of this data, such as mammography screenings can be collected and shared in information systems. The data can be used by medical personnel to increase their understanding of different diseases. It can also be used in computer aided diagnostics (CAD) where machine learning algorithms enable tools for intelligent data analysis. CAD makes use of machine learning techniques that learn a hypothesis, a statistical prediction about a patient's diagnosis from a large set of previously diagnosed examples.  The overarching purpose is to assist medical experts in more efficient and accurate diagnostics \parencite{li2007}. Machine learning algorithms is a well studied field within medical diagnosis and well suited for analyzing medical data, especially within small specialized diagnostic problems such as breast cancer \parencite{kononenko2001}.

Multiple studies of CAD on breast cancer have been conducted, primarily focusing on classifying mammography data of tumors as malignant or not, such those of \textcite{ramos2012} and \textcite{akay2009}. The act of feature selection, removing redundant or irrelevant features from a dataset, can provide classifiers to be faster, more cost-effective and accurate. With feature selection the understandability can be improved which is a clear benefit when it comes to medical decisions \parencite{karabulut2012}. It is also explicitly mentioned as a topic in need of more research in studies made on breast cancer diagnostics \parencite{akin2011}.


\section{Research Question}

In our thesis we will study the impact of four feature selection methods on the classification rate of malignant breast cancer by four different machine learning methods. We aim to answer the following:

\begin{itemize}
  \item Does the feature selection improve the accuracy of classification compared to using all features?
  \item Is the effect of feature selection dependent on the classification method?
\end{itemize}

Our hypothesis is that overall the feature selection will improve the classification rate of the machine learning methods used in the context of breast cancer classification. This hypothesis is based on previous research reported by \textcite{karabulut2012}, where it was found that classification accuracy on 15 different datasets of medical and non-medical data was increased by the use of filtering methods for feature selection.

Our research differs from the work presented in \parencite{karabulut2012} in the amount of datasets used and number of classification methods. In our project we will put more emphasis on breast cancer using only datasets of that type. The previous work only investigated feature selection methods by filtering which we will extend by implementing wrapper methods. Lastly, the research scope in this thesis includes a study of the effect of different feature selection methods on Support Vector Machines (SVMs), which was not included in \parencite{karabulut2012}.


\section{Approach}

Trials will be conducted with feature selection by using both wrapper methods and feature selection filter methods. The result of the feature selection methods will be used with different classifiers to evaluate their performance. To broaden the base for comparison we will use several classifiers, Decision Tree (DT) a logic based algorithms, Artificial Neural Network (ANN) a perceptron-based technique, Naïve Bayes (NB) a statistical learning algorithms and Support Vector Machines (SVM) \parencite{wallace2007}. These classifiers are commonly used in CAD and thus relevant to study \parencite{ramos2012}, \parencite{akay2009}, \parencite{li2007}. Feature selection (FS) methods and classifiers included in this report are denoted in table \ref{table:methods}.

\begin{table}[ht]
\begin{center}
\begin{tabular}{ l | l }
\multicolumn{2}{ c }{\textbf{Included classifiers and FS-methods}} \\
\hline
\multirow{4}{*}{Classifiers}
 & Artificial Neural Network (ANN) \\
 & Decision Tree (DT) \\
 & Naïve Bayes (NB) \\
 & Support Vector Machine (SVM) \\ \hline
\multirow{2}{*}{FS by Wrapping}
 & Sequential Backward FS (SBS) \\
 & Sequential Forward FS (SFS) \\ \hline
\multirow{2}{*}{FS by Filtering}
 & Chi-square (Chi2) \\
 & Entropy \\
\hline
\end{tabular}
\caption[]
{\small All classifiers and feature selection methods included in this paper. Each classifier will be tested with each FS-method yielding 16 distinct combinations.}
\label{table:methods}
\end{center}
\end{table}


A comparison between the classification rate of the machine learning methods without using any feature selection, and classification rate when using feature selection will be conducted. The comparison may then establish the importance of feature selection in different machine learning approaches when classifying breast cancer. The evaluation of impact by FS will be measured by computing the ratio between best accuracy achieved with and without using feature selection. Several datasets with different types of features will be used for evaluation. Details on the datasets will be presented in section \ref{sec:Datasets}. The reason for using multiple datasets is to strengthen the basis for statistical evaluation and possibility conclude a more generalized result.


\section{Scope}

The scope of this thesis is limited by the amount of datasets, classifiers and FS-methods included. Having four of each, the aim is to conclude a result that generalize well. However, it must be considered these are a small selection of all the available possibilities. The data represents a few of the possible ways to collect data when making breast cancer diagnostics. There are many more classifiers and each can be tuned into countless of configurations. Regarding the feature selection methods we evaluate two filter methods and two wrapper methods, it exists many more and also embedded methods which is not included in out thesis.

These limitations results in constraining our scope to the classifiers and FS-methods presented in table \ref{table:methods} and the datasets described in \ref{sec:Datasets}.


% --- 2. BACKGROUND CHAPTER ---
\chapter{Background}


\section{Computer Aided Diagnostics}

Machine learning techniques have been successfully applied to computer-aided diagnosis where it by a computerized procedure provides a second objective opinion for the assistance of medical image interpretation and diagnosis \parencite{li2007}, \parencite{ni2016}. To make a CAD system samples with diagnosis i used for learning. In the case of breast cancer diagnosis a radiologist put labels on a set of mammography scans. Labels that includes the diagnosis of the scan a possibly some attributes of the scan too. These scans togeather with the labels can then be used to learn a hypothesis whether a undiagnosed sample is benign or malignant cancer \parencite{li2007}.


\section{Breast Cancer}

A study in Sweden by \textcite{tabar2001} found breast carcinoma mortality was reduced by 63\% after mammography was introduced. This clearly emphasize the benefits of screening which had resulted in a increased usage of the method to detect and diagnose breast cancer. The increasing demand for mammography image interpretation lead to a shortage of medical radiologist to perform this task and consequently non medical personnel supplement the mammography image interpretation \parencite{culpan2016}. As breast cancer still continues to be the leading cause of cancer mortality among women and more efficient diagnostics and pathology is high on demand the need of low-cost point-of-care is very much needed as stated by \textcite{martei2018}.

Fine needle aspiration (FNA) is a diagnostic tool to aspirate cell samples by sampling cells, staining them and examine under a microscope \parencite{FNA}. An example of such sample can be seen in \ref{fig:fna_nuclei}. The cell samples can be evaluated within 24 hours and the method is cost-effective and can be used as a preoperative tool for investigation of tumors. The method is also complication-free and has been widely used for the past 60 years.

\begin{figure}[ht!]
  \centering
  \includegraphics[]{images/fna_nuclei.png}
  \caption{A caption of a FNA sample as seen through a microscope. Image courtesy of Wisconsin University}
  \label{fig:fna_nuclei}
\end{figure}

% Image source: http://webcache.googleusercontent.com/search?q=cache:635J2rrIdWsJ:pages.cs.wisc.edu/~olvi/uwmp/cancer.html+&cd=1&hl=sv&ct=clnk&gl=se


\section{Feature Selection}

A feature is a variable that describes a data instance. A rectangular surface can be considered having two features, length and height. A rectangular volume three features, length, height and depth. More complex data instances such as a gene expression may have up to 60,000 features and such a complex feature space results in a much harder learning process. Thus one often wishes to select a subset of all available to reduce the dimensionality \parencite{guyon2003}.

The benefits of selecting a subset of all available features are manyfold, among other it facilitates data visualisation and data understanding, reduces the measurement and storage requirements and reduces training and utilisation times. In cases with thousands of features like the example with gene expressions it is essential to work with a subset of the data to produce reliable results. \parencite{guyon2003}.


\subsection{Filter methods}

Filter methods are considered as a preprocessing step. That means a filter method evaluate features before data is applied to a learning machine or even before a deciding on a classifier. The evaluation is performed by doing variable ranking by some score such as information gain. The score results in a ranking of the attributes and a subset can be selected in order of the ranking \parencite{guyon2003}.

There are both good and bad aspects of filter methods. The positive concerns that variable ranking makes filter methods very scalable and robust as the calculations only operated on as many variables as there are features and can be performed just once and tested on multiple classifiers making it very computational effective. On the other hand, a subset of features that by their own might be assessed an non informative by their own may in combination provide a lot of information to enable good learning \parencite{guyon2003}.


\subsection{Wrapper methods}

\textbf{ Pawel: Add information on each classifier. At most 1/3 of a page describing their functionality on  vary high level }

Wrapper methods differs a lot from filter methods. While filter methods evaluate the as a preprocessing independent of the classifier, wrapper does the opposite.

Wrappers utilise the learning machine of interest as a black box to score subsets of variable according to their predictive power \parencite{guyon2003}. The issue is as datasets become very large this method might be overly computational intense as finding the optimal subset is considered to be NP-hard \parencite{amaldi1998}.


\section{Related Work}

\textbf{ Pawel: Make subsections where each section coves a specific topic. This can also be extended with almost another full page.}

The application of machine learning onto breast cancer classification is a well studied one. Since machine learning proved valuable empirical results breast cancer datasets have been widely used to assess the performance of a multitude of classification strategies and methods. [Insert ref. to early paper here].

Exhaustive studies for optimal Classifiers has been made as by \parencite{ramos2012} which tested 20,000 classification configurations to evaluated their ability to correctly classify malignant cancer. The achieved a result of 0.996 under the Receiver operating characteristic (ROC) curve.
% Add which method achieved the result above

With a foundation of well performing classifiers studies investigating more fine tuned approaches building upon earlier results were being made. \textcite{akin2011} demonstrated that ensemble learning can be used in CAD to improve the performance of rotation forest classifier. Using three different dataset and 30 classifying algorithms the average accuracy improved on all datasets by nearly 3\%.

\textcite{Abdel-Ilah2017} reported further improvements on ANNs by investigating the optimal number of hidden layers and neurons for a feed forward back propagation network. The highest accuracy acheived was 98\% using 3 hidden layers and 21 neurons with three distinct transfer functions.

\textcite{akay2009} investigated the performance of classification of a SVM with a RBF kernel using feature selection, filtering by F-score. They achieved a classification accuracy of 99.51\% which accordingly was among the highest scores recorded by then (2007).

\textcite{karabulut2012} made a comparative study on the effect of feature selection on classification accuracy and found up to 15.55\% improvement on classification rates. The study used only filter algorithms for feature selection, among those were both information and Chi-square. The study applied the selected features on three classification methods, Naive Bayes, Artificial Neural Network as Multilayer Perceptron, and J48 decision tree classifier on 15 different datasets including WBCD.

Building upon this foundation of work our study seeks to complete the field by providing further investigation of unreported selection methods such as SBS and SFS and review their performance on a combination of classifiers from the previous research presented here.


% --- 3. METHOD CHAPTER ---
\chapter{Method}

\section{Datasets}
\label{sec:Datasets}


\subsection{Wisconsin}

The dataset used in this thesis, Breast Cancer Wisconsin (Diagnostic) dataset, was donated 1995 to UCI  Machine Learning Repository \parencite{dua:2017} by one of its creators, Nick Street. It contains 569 instances with 32 attributes describing the features of breast cancer. Each instance is classified as benign (357) or malignant (212). The 32 attributes describe ten real-value features which are:

% Skip this extended description the dataset?
% \begin{itemize} \itemsep0pt \parskip0pt \parsep0pt
% 	\item \textbf{Radius:} Mean of distances from center to points on the perimeter.
% 	\item \textbf{Texture:} Standard deviation of gray-scale values.
% 	\item \textbf{Smoothness:} Local variation in radius lengths.
%   \item \textbf{Compactness:} perimeter\textsuperscript{2} / area - 1.
%   \item \textbf{Concavity:} Severity of concave portions of the contour.
%   \item \textbf{Concave points:} Number of concave portions of the contour.
%   \item \textbf{Fractal dimension:} Coastline approximation - 1.
%   \item \textbf{Perimeter:} Local variation in radius lengths.
%   \item \textbf{Area}
%   \item \textbf{Symmetry}
% \end{itemize}


\subsection{Royal Hallamshire Hospital}

Fine needle aspirates of breast lumps (FNAB) was collected from 692 patients at Royal Hallamshire Hospital, Sheffield, during 1992 - 1993. The FNABs 10 features of the FNABs was marked as present or non present. These features along with the patients's age defines the attributes of the dataset. In addition, the final outcome of benign disease or malignancy was confirmed by open biopsy where this result was available.

\subsection{MIAS database}

Mias database contain results from 119 data points with 5 features: Character of background tissue, Class of abnormality, X coordinate of centre of abnormality, Y coordinate of centre of abnormality, Approximate radius (in pixels). The features was extracted from 1024x1024 pixel images.

% Skip RNA dataset?
% \subsection{U.S. National Centre for Biotechnology Information}
%
% The data contains information on 1919 microRNAs in 122 different cases.

\subsection{Erlangen-Nuremberg}

Dataset collected from a Breast Imaging-Reporting and Data System (BI-RADS) at the Institute of Radiology of the University Erlangen-Nuremberg between 2003 and 2006. It contains three features assessed as a discrete value from a double-review by physicians along with the patients' age.

% Table with information on Datasets
\medskip
\begin{table}[ht!]
\begin{adjustwidth}{-5.in}{-5.in}
\begin{center}
   \begin{tabular}{l*{4}{l}}
   \hline
   Dataset         &
   \# examples  &
   \# features  &
   Ratio (B/M)     &
   Type            \\
   \hline
   Wisconsin (WBCD)						 &
   569                         &
   32                          &
   357/212                     &
   Continous                   \\
   Royal Hallamshire Hospital (RHH)  &
   692                         &
   11                          &
   457/235                     &
   Binary                      \\
   Erlangen-Nuremberg (EN)     &
   961                         &
   5                           &
   516/445                     &
   Discrete                    \\
   MIAS         							 &
   119                         &
   5                           &
   68/51                     	 &
   Discrete                    \\
  \hline
  \end{tabular}
  \caption{Datasets}
  \label{table:datasets_info}
\end{center}
\end{adjustwidth}
\end{table}


\section{Implementation}

\subsection{Classifiers}
All classifiers was implemented with Scikit \parencite{scikit-learn}. The parameters of every classifier was left to default. The parametrs default values are found in table \ref{table:classifier_params}.

\begin{table}[ht]
\begin{center}
\begin{tabular}{ l | r | l }
\multicolumn{3}{ c }{\textbf{Classifier parameters}} \\
\hline
\multirow{1}{*}{ANN: Multi-layer Perceptron}
  & Hidden layers & 2 \\
  & Layer size & 100 \\
  & Activation & ReLU \\
  & Solver & Adam \\ \hline
\multirow{1}{*}{Decision Tree: CART}
  & Criterion & Gini \\
  & Splitter & Best \\
  & Max depth & None \\ \hline
\multirow{1}{*}{Naive Bayes}
  & Type & Gaussian \\
  & Priors & None \\ \hline
\multirow{1}{*}{SVM}
  & Kernel & Rbf \\
  & Degree & 3 \\
  & Penalty & 1.0 \\ \hline
\end{tabular}
\caption{Parameters of each classifier used}
\label{table:classifier_params}
\end{center}
\end{table}


\subsection{Feature selection}

The feature selection consisted of two parts. First, the filter methods which was implemented with the \textit{Select K best} method, provided by Scikit \parencite{scikit-learn}. Second, the wrapper methods which was implemented with the \textit{SequentialFeatureSelector} method, provided in mlextend library \parencite{mlextend}.


\section{Evaluation}

\textbf{ Pawel: Add formulas on how we evaluate and how results are produces. Also, section on ANOVA. }

During experiments each dataset is loaded and split into training and test data with a ratio of 3:1. The same training and test data is used for every classifier. Each classifier is tested iteratively with each filter method. The accuracy is the computed mean of 10 fold cross validation on the test set.

% Pawel: Describe all configurations of classifiers and feature selection methods, discuss how you choose parameters, train them (on what data with what algorithms etc.)


% --- 4. RESULTS AND ANALYSIS CHAPTER ---
\chapter{Results and Analysis}

We performed classification with four different classifiers on four different datasets. In each dataset-classifier combination we measured classification accuracy when applying four different feature selection (FS) methods. The aim is to investigate the impact of feature selection on classification accuracy, and the interaction between different classifiers and FS-methods. To enable a basis for comparisons, classification accuracy was measured without FS too.

\section{Impact of Factors: Datasets, Classifiers and FS-methods}
\label{Variation_among_factors}

Before analyzing the results we consider potential interactions between different factors involved in the experiment, these are:

\begin{enumerate}
  \item Datasets
  \item Machine Learning classifiers
  \item Feature selection methods
\end{enumerate}

Having four of each factor, there are 16 distinct combinations of classifiers and FS-methods with four measurements, one on each dataset. All datasets contain different information and therefore introduce necessary variance to the experiments. The variance allows us to suspect these results generalizes well onto other datasets in the domain of breast cancer, which are not included in this study. In our scope we are particular interested in the interaction of classifiers and FS-methods.

% \begin{figure*}[ht]
  \centering
    \begin{subfigure}[b]{0.475\textwidth}
        \centering
        \includegraphics[width=\textwidth]{../plots_with_std_fill/comp_acc_datasets.png}
        \caption[]%
        {{\small Datasets and accuracy displays positive correlation independently of applied FS method.}}
        \label{fig:comp_acc_datasets}
    \end{subfigure}
  \hfill
    \begin{subfigure}[b]{0.475\textwidth}
        \centering
        \includegraphics[width=\textwidth]{../plots_with_std_fill/comp_classif_datasets.png}
        \caption[]%
        {{\small Some correlation between different classifiers and FS-methods is evident.}}
        \label{fig:comp_classif_datasets}
    \end{subfigure}
  \caption[]
  {\small Correlation between factors of the experiments.}
  \label{fig:anova_plots}
\end{figure*}


\subsection{Evaluation of FS-methods and Classifiers}
\label{sec:fs_methods_classifiers}

To investigate the effect of feature selection on different classifiers, we performed two-way ANOVA. The values constitutes of four measurements by each of the 16 distinct classifier and FS-method combination. The four measurements come from each of the four datasets. Each measurement is the best accuracy achieved with a attribute subset of the dataset. The ANOVA result is presented in table \ref{table:anova_values_classif}.

\begin{table}[ht]
  \begin{center}
  \begin{tabular}{l|r|r|r|r|l}
  \cline{2-5}
  & $RSS$ & $df$ & $F$ & $P(>F)$ \\ \cline{1-5}
  \multicolumn{1}{ |l| }{\textbf{Classifier}}
  & 0.3336 &  3.0 & 4.660   & 0.00615    & ** \\
  \cline{1-5}
  \multicolumn{1}{ |l| }{\textbf{Method}}
  & 0.0099 &  3.0 & 0.138   & 0.93666 & \\
  \cline{1-5}
  \multicolumn{1}{ |l| }{\textbf{Classifier:Method}}
  & 0.0631 &  9.0 & 0.294   & 0.97314 \\
  \cline{1-5}
  \multicolumn{1}{ |l| }{\textbf{Residual}}
  & 1.1455 &  48.0 \\ \cline{1-3}
  \end{tabular}
  \caption[]%
  {{\small ANOVA values of accuracy in relation to classifier, method and the interaction of classifiers and methods. $RSS$: Residual sum of squares. $df$: Degrees of Freedom. $F$: Mean Square for the Model divided by the Mean Square for Error (error/residual).  $P(>F)$: the significance probability associated with $F$. The stars indicate the range of significant level: 0 "***" 0.001 "**" 0.01 "*" 0.05 " . " 0.1 " " 1.}}
  \label{table:anova_values_classif}
  \end{center}
\end{table}


The **-significance of Classifier concludes that the selection of classifier effect what accuracy is achieved. The variance of the each respective classifier is visualized in figure \ref{fig:box_plots}. The box-plot entail that ANN on average perform worse than other classifiers. NB has the largest variance, with accuracy ranging from roughly 50\% to 100\%. CART performs the best, with the highest average accuracy.

The ANOVA result of Method conclude the selection of FS-method do not result in a significant difference in achieved accuracy. In figure \ref{fig:box_methods} all methods manifests a large variance but roughly in the same range, which emphasizes the result of the ANOVA. That is, there is no significant difference in accuracy when applying different FS-methods.

Analyzing these results together, first they suggest that the selection of classifier impacts the expected accuracy. Secondly, which FS-method is used together with the classifier do not effect expected accuracy. Lastly, no unique combination of classifier and FS-method are statistically superior to others.

It's important to note these measurements only include classification with FS-selection, they do not enable any conclusions regarding the impact of using, versus not using feature selection. Therefore, that is the next point of interest.

\begin{figure*}[ht]
  \centering
    \begin{subfigure}[b]{0.49\textwidth}
        \centering
        \includegraphics[width=\textwidth]{Rplots/boxplot_classifiers.png}
        \caption[]%
        {{\small}}
        \label{fig:box_classifers}
    \end{subfigure}
  \hfill
    \begin{subfigure}[b]{0.49\textwidth}
        \centering
        \includegraphics[width=\textwidth]{Rplots/boxplot_methods.png}
        \caption[]%
        {{\small}}
        \label{fig:box_methods}
    \end{subfigure}
  \caption[]
  {\small Variation in accuracy in respect to (a) classifiers and (b) FS-methods. Vertical axis represents accuracy. The horizontal bar in each box represents the mean accuracy. The box represents the range where 50\% of the measurements was found. The vertical lines connected to the boxes includes represents the remaining 25\% on each side. Dots represents outliers.}
  \label{fig:box_plots}
\end{figure*}


\FloatBarrier
\section{Classification Improvements}

After performing classification with and without FS-methods, all results of each classifier was collected. Results are presented in tables for ANN \ref{table:ANN}, CART \ref{table:CART}, NB \ref{table:NB} and SVM \ref{table:SVM} where the highest achieved accuracy is highlighted in bold format. The improvement is measured in gain, the ratio between best achieved FS accuracy and full dataset accuracy.

% --- Tables ---
\begin{table}[!htb]

    \noindent\makebox[1.05\textwidth]{%

      \begin{subtable}{0.73\linewidth}
        \centering
          \begin{tabular}{|l|l|l|l|l|l}
\cline{1-5}
        & MIAS              & EN                & RHH               & WBCD      &         \\
\cline{1-5}
Chi2    & \textbf{0.63333}  & 0.55357           & 0.66993           & \textbf{0.72810}  &         \\
Entropy & 0.56667           & \textbf{0.83214}  & \textbf{0.90196}  & 0.63762           &         \\
SBS     & 0.50000           & 0.55214           & 0.66144           & 0.64286           &         \\
SFS     & 0.53333           & 0.50595           & 0.65915           & 0.67381           &         \\
Full    & 0.60000           & 0.52714           & 0.57320           & 0.64143 \\
\cline{1-5}
\cline{1-5}
Gain    & 0.05555           & 0.57859           & 0.57355           & 0.13512           & 1.34281 \\
\cline{1-5}
\end{tabular}

          \caption[]
          \small{}
          \label{table:ANN}
      \end{subtable}%
      \begin{subtable}{0.73\linewidth}
        \centering
          \begin{tabular}{|l|l|l|l|l|l}
\cline{1-5}
        & MIAS              & EN                & RHH               & WBCD      &         \\
\cline{1-5}
Chi2    & 0.70000           & 0.69262           & 0.90196           & \textbf{0.96524}  &         \\
Entropy & 0.53333           & \textbf{0.83214}  & 0.91373           & 0.95905           &         \\
SBS     & 0.63333           & 0.67286           & 0.91307           & 0.95143           &         \\
SFS     & 0.66667           & 0.67286           & \textbf{0.91895}  & \textbf{0.96524}  &         \\
Full    & \textbf{0.76667}  & 0.68786           & 0.89608           & 0.93714           &         \\
\cline{1-5}
\cline{1-5}
Gain    & -0.0870           & 0.20975           & 0.02552           & 0.02998           & 0.17825 \\
\cline{1-5}
\end{tabular}

          \caption[]
          \small{}
          \label{table:CART}
      \end{subtable}
    }

    \vskip
    \baselineskip

    \noindent\makebox[1.05\textwidth]{%

      \begin{subtable}{0.73\linewidth}
        \centering
          \begin{tabular}{|l|l|l|l|l|l}
\cline{1-5}
        \textbf{NB} & MIAS              & EN                & RHH               & WBCD      &         \\
\cline{1-5}
Chi2    & \textbf{0.75} &  0.75 &  0.93 & 0.97 \\
Entropy & 0.72 &  0.55 &  0.93 & \textbf{0.98} \\
SBS     & 0.51 &  0.72 &  0.93 & 0.97 \\
SFS     & 0.51 &  0.72 &  0.93 & 0.97 \\
Full    & 0.57 &  \textbf{0.93} &  \textbf{0.96} & 0.96 \\
\cline{1-5}
Gain    & 0.30 & -0.19 & -0.03 & 0.02 & 0.1\\
\cline{1-5}
\end{tabular}

          \caption[]
          \small{}
          \label{table:NB}
      \end{subtable}%
      \begin{subtable}{0.73\linewidth}
        \centering
          \begin{tabular}{|l|l|l|l|l|l}
\cline{1-5}
        \textbf{SVM} & MIAS              & EN                & RHH               & WBCD      &         \\
\cline{1-5}
Chi2    &  0.59 &  0.75 & 0.92 & 0.66 \\
Entropy &  0.59 &  0.84 & 0.92 & 0.66 \\
SBS     &  0.59 &  0.75 & \textbf{0.94} & \textbf{0.94} \\
SFS     &  0.56 &  0.75 & 0.90 & \textbf{0.94} \\
Full    &  \textbf{0.74} &  \textbf{0.90} & 0.64 & 0.64 \\
\cline{1-5}
Gain    & -0.20 & -0.07 & 0.46 & 0.45 & 0.64 \\
\end{tabular}

          \caption[]
          \small{}
          \label{table:SVM}
      \end{subtable}
    }

    \caption[]
    {\small
      Mean accuracy of 10-fold cross-validation accuracies achieved on each dataset when applying some FS-method or with Full dataset. Tables are categorized by classifier; (a) ANN, (b) CART Decision tree, (c) Na\"ive Bayes and (d) Support Vector Machine. The highest accuracy is highlighted by bold font and Gain represents ratio between the best accuracy achieved by FS and the accuracy by Full dataset. Accumulated Gain is last on the Gain row. Best accuracy by FS is equivalent or greater than with Full dataset in every instance except in (b) with CART classifier on MIAS dataset.
    }
    \label{table:combines_tables}
\end{table}


\FloatBarrier
\subsection{Classification Improvements' Significance}
\label{sec:Investigation_improvement}

To study the effect of feature selection on classification accuracy we perform a t-test. The t-test values constitutes of the accuracies achieved without feature selection as one distribution. The other are the best accuracies achieved with feature selection. The t-test values are summarized in \ref{table:t_test_values}.

\begin{table}[!htb]

    \noindent\makebox[\textwidth]{%

      \begin{subtable}{0.65\linewidth}
        \centering
          \begin{tabular}{|l|l|l|l|l|l}
          \cline{1-5}
          \textbf{FS}      & MIAS           & EN             & RHH            & WBCD           &         \\
          \cline{1-5}
          ANN     & 0.63           & 0.83           & 0.90           & 0.73           &         \\
          CART    & 0.67           & 0.83           & 0.92           & 0.97           &         \\
          NB      & 0.77           & 0.74           & 0.94           & 0.97           &         \\
          SFS     & 0.57           & 0.83           & 0.91           & 0.93           &         \\
          \cline{1-5}
          \end{tabular}
          \caption[]
          \small{}
      \end{subtable}%
      \begin{subtable}{0.65\linewidth}
        \centering
          \begin{tabular}{|l|l|l|l|l|l}
          \cline{1-5}
          \textbf{Full}    & MIAS           & EN             & RHH            & WBCD           &         \\
          \cline{1-5}
          ANN     & 0.60           & 0.53           & 0.57           & 0.64           &         \\
          CART    & 0.77           & 0.69           & 0.90           & 0.94           &         \\
          NB      & 0.77           & 0.66           & 0.94           & 0.96           &         \\
          SFS     & 0.57           & 0.73           & 0.90           & 0.61           &         \\
          \cline{1-5}
          \end{tabular}
          \caption[]
          \small{}
      \end{subtable}
    }

    \caption[]%
    {\small
      (a) The highest accuracy achieved by any FS-method on all classifier-dataset combinations. (b) corresponding accuracies achieved without any feature selection on Full dataset. T-test comparing the distributions results in significantly higher results by classification with feature selection.}
    \label{table:t_test_values}

\end{table}


Comparing the t-test results of all classifiers, t-stat value suggests a 28\% increased performance when applying feature selection, see table \ref{table:ttest_result}. However, significance is insufficient to reject the null hypothesis that distributions are equal. Consequently we can not conclude feature selection significantly improves classification accuracy based on our data.

To further analyze the differences, t-test is performed to compare each accuracy by each individual classifier, with and without feature selection. The tests concluded a significant improvement of accuracy using FS on ANN. No significant increase nor decrease in accuracy could be proven for the other classifiers; DT, CART and SVM.



\begin{table}[htbp]
\begin{center}
\begin{tabular}{|l|c|c|c}
\cline{1-3}
\textbf{Classifier} & \textbf{t-stat} & \textbf{P}\\
\cline{1-3}
          All &   1.28 &   0.21 &    \\
          ANN &   2.55 &   0.04  &  * \\
          SVM &   0.95 &   0.38 &   \\
          NB &   0.04 &   0.97 &     \\
          CART &   -0.43 &  0.68 &   \\
\cline{1-3}
\end{tabular}
\caption[]
{\small
  Significant values of t-test on Classifiers.  t-stat is a ratio between the difference between two groups and the difference within the groups. P: Significance probability. The stars indicate the range of significant level: 0 "***" 0.001 "**" 0.01 "*" 0.05 " . " 0.1 " " 1.
}
\label{table:ttest_result}
\end{center}
\end{table}


%
% --- Student t-test [FULL] ---
% t-stat:  1.277837907846637
%  prob:  0.21110632421181863
% --- Student t-test [ANN] ---
% t-stat:  2.5510324322268363
%  prob:  0.043430149329747425
% --- Student t-test [CART] ---
% t-stat:  -0.4377851091863364
%  prob:  0.676862029454676
% --- Student t-test [NB] ---
% t-stat:  0.04364357804719851
%  prob:  0.9666047126659922
% --- Student t-test [SVM] ---
% t-stat:  0.9474693713881566
%  prob:  0.379979860939623


The absence in the significance among the t-test results may be a consequence of data shortage. However, t-stat indicates some differences among classifiers that we'll analyze further.

\FloatBarrier
\subsection{Differences among Classifiers}

We found in \ref{sec:fs_methods_classifiers} that the selection of classifier affect the expected accuracy. In \ref{sec:Investigation_improvement} t-test scores concluded benefit of feature selection varies between different classifiers. Ranking the accumulated gain of each classifier from tables \ref{table:combines_tables}, we construct table \ref{table:gain_comparison} which clarifies variation of improved accuracy between classifiers. We'll now look at each classifier in turn:

\begin{table}[hp]
  \begin{tabular}{|l|l|l|}
\hline
Ranking  & Classifier                & Accumulated gain  \\
\hline
1        & Artificial Neural Network & 1.34              \\
2        & Support Vector Machine    & 0.68              \\
3        & Decision Tree             & 0.18              \\
4        & Naive Bayes               & 0.14              \\
\hline
\end{tabular}

  \caption[]%
  {{\small Ranking of which classifiers gained most accuracy when comparing feature selection to full dataset.}}
  \label{table:gain_comparison}
\end{table}

\subsubsection{Artificial Neural Network}

Looking at the table \ref{table:gain_comparison} the accumulated gain was 1.2 which was the highest among all classifiers. However, ANN consistently performs the worst of all classifiers in terms of accuracy. The ANN also provides the least consistent results with strong fluctuations in the results and wide standard deviation margins. Such fluctuations may suggest issues regarding convergence as an effect of using default parameters in the network. These characteristics are evident in plot \ref{fig:ANN_WBCD} showcasing the accuracy by each FS-method as a function of number of attributes. However, t-test show an significant increase in classification accuracy when feature selection is applied.

\subsubsection{Support Vector Machine}

SVM improves accuracy in two out of four datasets in table \ref{table:SVM}. In these datasets the best performance is achieved by using wrapper methods.

The SVM behaves differently in respect to each dataset which may conclude different kernels of the SVM is needed on different datasets. Improvements are seen with a larger subset of attributes in the EN dataset \ref{fig:SVM_EN}. A negative trend on accuracy is observed on the WBCD dataset \ref{fig:SVM_WBCD} which might indicate an issue with dimensionality. In \ref{fig:SVM_RHH} an improved accuracy is evident with maximal accuracy achieved on a subset suggesting a positive effect of feature selection.

\subsubsection{Decision Tree}

The decision tree shows consistent performance, generally increasing accuracy with an increased amount of features. Although in cases like plot \ref{fig:CART_MIAS} and \ref{fig:CART_EN} best accuracy is achieved with a subset of features displaying evident benefits of feature selection.

\subsubsection{Naïve Bayes}

In plot \ref{fig:NB_WBCD} the accuracy presents little to no improvement when increasing the number of attributes. In three out of four datasets NB produces the highest accuracy with very few attributes indicating benefits of feature selection.


\begin{figure*}[!htbp]

  \noindent\makebox[1.2\textwidth]{
    \begin{subfigure}[b]{0.7\textwidth}
        \centering
        \includegraphics[width=\textwidth]{../plots_with_std_fill/ANNd1.png}
        \caption[]%
        {{\small}}
        \label{fig:ANN_MIAS}
    \end{subfigure}
    \hfill
    \begin{subfigure}[b]{0.7\textwidth}
        \centering
        \includegraphics[width=\textwidth]{../plots_with_std_fill/ANNd2.png}
        \caption[]%
        {{\small}}
        \label{fig:ANN_EN}
    \end{subfigure}
    }
    \vskip
    \baselineskip
    \noindent\makebox[1.2\textwidth]{
    \begin{subfigure}[b]{0.7\textwidth}
        \centering
        \includegraphics[width=\textwidth]{../plots_with_std_fill/ANNd3.png}
        \caption[]%
        {{\small}}
        \label{fig:ANN_RHH}
    \end{subfigure}
    \quad
    \begin{subfigure}[b]{0.7\textwidth}
        \centering
        \includegraphics[width=\textwidth]{../plots_with_std_fill/ANNd4.png}
        \caption[]%
        {{\small}}
        \label{fig:ANN_WBCD}
    \end{subfigure}
    }

    \caption[]
    {\small
      Classifier ANN. Each plot corresponds to datasets (a) MIAS, (b) EN, (c) RHH and (d) WBCD. x-axis is number of features, y-axis is mean accuracy achieved by corresponding feature selection method. Shaded area represent the standard error for each FS-method.
    }
    \label{fig:plots_ANN}

\end{figure*}

\begin{figure*}[htbp]
  \noindent\makebox[1.1\textwidth]{
  \begin{subfigure}[b]{0.7\textwidth}
      \centering
      \includegraphics[width=\textwidth]{../plots_with_std_fill/CARTd1.png}
      \caption[]%
      {{\small}}
      \label{fig:CART_MIAS}
  \end{subfigure}
  \hfill
  \begin{subfigure}[b]{0.7\textwidth}
      \centering
      \includegraphics[width=\textwidth]{../plots_with_std_fill/CARTd2.png}
      \caption[]%
      {{\small}}
      \label{fig:CART_EN}
  \end{subfigure}

  }
  \vskip
  \baselineskip
  \noindent\makebox[1.1\textwidth]{

  \begin{subfigure}[b]{0.7\textwidth}
      \centering
      \includegraphics[width=\textwidth]{../plots_with_std_fill/CARTd3.png}
      \caption[]%
      {{\small}}
      \label{fig:CART_RHH}
  \end{subfigure}
  \hfill
  \begin{subfigure}[b]{0.7\textwidth}
      \centering
      \includegraphics[width=\textwidth]{../plots_with_std_fill/CARTd4.png}
      \caption[]%
      {{\small}}
      \label{fig:CART_WBCD}
  \end{subfigure}
  }
  \caption[]
  {\small
    Classifier CART. Each plot corresponds to datasets (a) MIAS, (b) EN, (c) RHH and (d) WBCD. x-axis is number of features, y-axis is mean accuracy achieved by corresponding feature selection method. Shaded area represent the standard error for each FS-method.
  }
  \label{fig:plots_CART}
\end{figure*}

\begin{figure*}[htbp]
  \centering
  \begin{subfigure}[b]{0.475\textwidth}
      \centering
      \includegraphics[width=\textwidth]{../plots_with_std_fill/NBd1.png}
      \caption[]%
      {{\small}}
      \label{fig:NB_MIAS}
  \end{subfigure}
  \hfill
  \begin{subfigure}[b]{0.475\textwidth}
      \centering
      \includegraphics[width=\textwidth]{../plots_with_std_fill/NBd2.png}
      \caption[]%
      {{\small}}
      \label{fig:NB_EN}
  \end{subfigure}

  \begin{subfigure}[b]{0.475\textwidth}
      \centering
      \includegraphics[width=\textwidth]{../plots_with_std_fill/NBd3.png}
      \caption[]%
      {{\small}}
      \label{fig:NB_RHH}
  \end{subfigure}
  \quad
  \begin{subfigure}[b]{0.475\textwidth}
      \centering
      \includegraphics[width=\textwidth]{../plots_with_std_fill/NBd4.png}
      \caption[]%
      {{\small}}
      \label{fig:NB_WBCD}
  \end{subfigure}
  \caption[]
  {\small
    Classifier NB. Each plot corresponds to datasets (a) MIAS, (b) EN, (c) RHH and (d) WBCD. x-axis is number of features, y-axis is mean accuracy achieved by corresponding feature selection method.
  }
  \label{fig:plots_NB}
\end{figure*}

\begin{figure*}[htbp]
  \noindent\makebox[1.1\textwidth]{
  \begin{subfigure}[b]{0.7\textwidth}
      \centering
      \includegraphics[width=\textwidth]{../plots_with_std_fill/SVMd1.png}
      \caption[]%
      {{\small}}
      \label{fig:SVM_MIAS}
  \end{subfigure}
  \hfill
  \begin{subfigure}[b]{0.7\textwidth}
      \centering
      \includegraphics[width=\textwidth]{../plots_with_std_fill/SVMd2.png}
      \caption[]%
      {{\small}}
      \label{fig:SVM_EN}
  \end{subfigure}
  }
  \vskip
  \baselineskip
  \noindent\makebox[1.1\textwidth]{
  \begin{subfigure}[b]{0.7\textwidth}
      \centering
      \includegraphics[width=\textwidth]{../plots_with_std_fill/SVMd3.png}
      \caption[]%
      {{\small}}
      \label{fig:SVM_RHH}
  \end{subfigure}
  \quad
  \begin{subfigure}[b]{0.7\textwidth}
      \centering
      \includegraphics[width=\textwidth]{../plots_with_std_fill/SVMd4.png}
      \caption[]%
      {{\small}}
      \label{fig:SVM_WBCD}
  \end{subfigure}
  }
  \caption[]
  {\small
    Classifier SVM. Each plot corresponds to datasets (a) MIAS, (b) EN, (c) RHH and (d) WBCD. x-axis is number of features, y-axis is mean accuracy achieved by corresponding feature selection method. Shaded area represent the standard error for each FS-method.
  }
  \label{fig:plots_SVM}
\end{figure*}


\newpage
\section{Computation time}
\label{sec:cumtime}

Profiling the execution of running all experiments approximately 100\% of CPU-time was allocated to the Wrapper algorithms as showed in table \ref{table:cpu}. Finding the best possible subset of features is considered a NP-hard problem meaning a solution can not be found in polynomial time. This clearly suggest favouring filtering methods when choosing a feature selection method having limited computational resources.

ncalls  tottime  percall  cumtime  percall filename:lineno(function)

188    0.006    0.000   42.802    0.228 Main1_profile.py:103(entropyFunction)
  1    0.005    0.005 28647.116 28647.116 Main1_profile.py:11(<module>)
188    0.002    0.000 24785.653  131.839 Main1_profile.py:111(sfsFunction)
188    0.003    0.000 3751.879   19.957 Main1_profile.py:119(sbsFunction)
752    0.033    0.000    0.045    0.000 Main1_profile.py:127(subsetFrom)
 16    0.061    0.004 28646.085 1790.380 Main1_profile.py:131(evaluateMethod)
  1    0.002    0.002 28646.236 28646.236 Main1_profile.py:35(main)
  4    0.000    0.000    0.002    0.001 Main1_profile.py:92(splitData)
188    0.005    0.000    3.759    0.020 Main1_profile.py:95(chi2Function)



% --- 5. DISCUSSION ---
\chapter{Discussion}

[ Remind the reader of key finding ]

\section{Influence of feature selection}

Feature selection is valuable strategy when it comes to classification of breast cancer. As stated in the results all instances but one provided equally good or improved accuracy on the dataset with feature selection compared to using the full dataset. Even in cases where results are equally good feature selection still provide benefits to the process. The benefits consists of potentially lessened effort gathering and processing less attributes at data collection. As stated by \textcite{martei2018} there is a large demand of a more streamlined and efficient process when it comes to breast cancer classification, feature selection seems to offer benefits to include such process.

Our findings suggests choosing the correct combination of classifier and FS-method is essential. While some feature selection method nearly always outperform the classifier with the full dataset it does not apply to all FS methods. As filter methods proved computationally fast in comparison to wrappers, suitable classifiers can be investigated efficiently to find a good subset of features with filter methods.

If wrapper methods are to be used there may be a large benefit in constructing a search approach for the NP-hard part of the problem instead of evaluating the full search space. Such a study has been conducted by \textcite{panthong2015} and improved both classification accuracy and reduced runtime.

As mentioned in source of errors \ref{sec:source_of_errors} using default classifiers might cause the fluctuation and wide variance in the results. A classifier with $n$ attributes might need very different parameters than the same classifier with $n + 1$ parameters to achieve optimal performance. This raises an important question: should a high performing classifier and FS-method be found first then optimized by tuning or it it the other way around?

\subsection{Comparing classification accuracy}

Reports that achieve high classification accuracy such as \textcite{akay2009} decide on one classifier and FS-method and optimize the parameters of the classifier to both the FS-method and the dataset. It results in high performance but leaves the question how such classifier and FS combinations should be chosen.

On average FS improved the result of the ANN classifier by 30\%. This result can be compared with the 28\% gain achieved by \textcite{karabulut2012}. They similar result can be explain by the similar classifiers and the use of different filter method not having an high impact for ANN. The SVM classifier received an average gain of 16\%. Comparing this results with \textcite{b20103177} our result shows a greater impact of FS using a SVM classifier. This is probably an effect of different datasets and FS methods being used.

\section{Limitations}

Due to the limited amount of breast cancer datasets found and utilized in the study its difficult to confidently draw conclusions regarding all breast cancer classification at large.

Limited resources has also resulted in a reduced subset of FS-methods studied. While having two methods of each FS-family, filters and wrappers, there are many more which may have produced different results than we have achieved.


\section{Ethical aspects}

The best classification accuracy found in this report was 97\%. Studies show machine learning already outperforms medical experts in setting a correct diagnose of breast cancer \parencite{fnab}. When diagnostics progresses from being made by humans to machines many factors need to be considered. Do humans trust computers enough to allow this progression, should they even be informed their diagnose is set by algorithms? If so, how can we explain a certain output when many algorithms are truly hard to interpret. Lastly, what data should be used for training, only collecting data of those who have access to such healthcare may introduce a bias against other demographics of the population.

The data used in this report origin from real patients that may be experience discomfort during mammographies, FNA sampling or with other method was used when collecting the data. The data also holds sensitive information ruling the patients future health. Data is to our knowledge never collected without a patients consent and carries no information that can allows any identification of the patient.

\section{Sustainability}

We trust the reliability in our findings and believe they contribute to the accumulated knowledge of the field as they are made available. In that sense the of classification and breast cancer research progresses forward and can in turn make new discoveries that enables a more sustainable future.


% --- 6. CONCLUSION ---
\chapter{Conclusion}

% Address research questions here;

% 1. Does the feature selection improve the accuracy of classification compared to using all features?
% 2. In which machine learning methods does feature selection have the greatest impact?

Applying feature selection methods provides an improved classification accuracy on benign or malignant breast cancer. The result was consistent over a multitude of classifiers, feature selection methods and datasets. The improved accuracy was dependent on a classifier and feature selection method used, as well as which dataset being used on. Results indicate one needs to select these variables carefully to see improvements and further investigation on such methodology is needed.

The machine learning that overall benefited most from feature selection in terms of accuracy was Artificial neural network. On 4 different datasets with significantly different compositions in all cases accuracy was improved. The larges improvement was 58\% when applying feature selection by entropy on the Erlangen-Nuremberg dataset.



% --- BIBLIOGRAPHY AND APPENDIX ---

\printbibliography[heading=bibintoc]

% Print the bibliography (and make it appear in the table of contents)

\appendix

\chapter{Appended Material}

\begin{table}[ht]
\begin{center}
\begin{tabular}{ l | r | l }
\multicolumn{3}{ c }{\textbf{Classifier parameters}} \\
\hline
\multirow{1}{*}{ANN: Multi-layer Perceptron}
  & Hidden layers & 2 \\
  & Layer size & 100 \\
  & Activation & ReLU \\
  & Solver & Adam \\ \hline
\multirow{1}{*}{Decision Tree: CART}
  & Criterion & Gini \\
  & Splitter & Best \\
  & Max depth & None \\ \hline
\multirow{1}{*}{Naive Bayes}
  & Type & Gaussian \\
  & Priors & None \\ \hline
\multirow{1}{*}{SVM}
  & Kernel & Rbf \\
  & Degree & 3 \\
  & Penalty & 1.0 \\ \hline
\end{tabular}
\caption{Parameters of each classifier used}
\label{table:classifier_params}
\end{center}
\end{table}


\end{document}
