\chapter{Discussion}

\section{Influence of feature selection}

Feature selection is valuable strategy when it comes to classification of breast cancer. As stated in the results all instances but one provided equally good or improved accuracy on the dataset with feature selection compared to using the full dataset. Even in cases where results are equally good feature selection still provide benefits to the process. The benefits consists of potentially lessened effort gathering and processing less attributes at data collection. As stated by \textcite{martei2018} there is a large demand of a more streamlined and efficient process when it comes to breast cancer classification, feature selection seems to offer benefits to include such process.

Our finding suggests choosing the correct combination of classifier and FS-method is essential to produce a high accuracy. While some feature selection method nearly always outperform the classifier with the full dataset it does not apply to all FS methods. As filter methods proved computationally fast in comparison to wrappers, suitable classifiers can be investigated efficiently to find a good subset of features with filter methods.

If wrapper methods are to be used there may be a large benefit in constructing a search approach for the NP-hard part of the problem instead of evaluating the full search space. Such a study has been conducted by \textcite{panthong2015} and improved both classification accuracy and reduced runtime.

As mentioned in source of errors \ref{sec:source_of_errors} using default classifiers might cause the fluctuation and wide variance in the results. A classifier with $n$ attributes might need very different parameters than the same classifier with $n + 1$ parameters to achieve optimal performance. This raises an important question: should a high performing classifier and FS-method be found first then optimized by tuning or it it the other way around?

Reports that achieve high classification accuracy such as \parencite{akay2009} decide on one classifier and FS-method and optimize the parameters of the classifier to both the FS-method and the dataset. It results in high performance but leaves the question how such classifier and FS combinations should be chosen.


\section{Further research}

Further investigation of a methodology of finding the best possible classifier and FS-method.

More recent strategies of diagnosing breast cancer involves sampling microRNA from patients. Other diseases have been diagnosed with microRNA suggesting promising results of the method. As a sample of microRNA contains around 2 000 features selection may offer a huge benefit.


\section{Limitations}

Due to the limited amount of breast cancer datasets found and utilized in the study its difficult to confidently draw conclusions regarding all breast cancer classification at large.

Limited resources has also resulted in a reduced subset of FS-methods studied. While having two methods of each FS-family, filters and wrappers, there are many more which may have produced different results than we have achieved.


\section{Ethical aspects}

The data used in this report origin from patients that may be experience discomfort during mammographies, FNA sampling or with other method was used when collecting the data. The data also holds sensitive information ruling the patients future health.

While circumstances of a patient can be very unpleasant such test is a tool to diagnose the patient and potentially discover cancer in time to take necessary action to improve patients' health. Data is to our knowledge never collected without a patients consent and carries no information that can allows identification.

\section{Sustainability}

Under construction...
