\chapter{Conclusion}

% Address research questions here;

% 1. Does the feature selection improve the accuracy of classification compared to using all features?


Applying feature selection methods provides an improved classification accuracy on benign or malignant breast cancer. The result was consistent over a multitude of classifiers, feature selection methods and datasets. The improved accuracy was dependent on a classifier and feature selection method used, as well as which dataset being used on. Results indicate one needs to select these variables carefully to see improvements and further investigation on such methodology is needed.

% 2. In which machine learning methods does feature selection have the greatest impact?

The machine learning that overall benefited most from feature selection in terms of accuracy was Artificial neural network. On four different datasets with significantly different compositions in all cases accuracy was improved. The larges improvement was 58\% when applying feature selection by entropy on the Erlangen-Nuremberg dataset.


\section{Further research}

Further investigation of a methodology of finding the best possible classifier and FS-method.

More recent strategies of diagnosing breast cancer involves sampling microRNA from patients. Other diseases have been diagnosed with microRNA suggesting promising results of the method. As a sample of microRNA contains around 2 000 features, selection may offer a huge benefit in line with our findings that a increased number of feature benefits more from feature selection.
